\chapter*{Glossaire}

\begin{itemize}
    \item \textbf{Algorithme :} Une suite d'instructions ou d'opérations permettant de résoudre un problème.
    \item \textbf{Arbre :} Entité mathématique représentant un graphe avec un nœud parent et plusieurs enfants.
    \item \textbf{Arc (graphe) :} Entité mathématique représentant un lien entre deux nœuds.
    \item \textbf{API :} Application Programming Interface. Il s'agit d'une interface par laquelle
il est possible d'échanger des informations avec une entité informatique.
    \item \textbf{API REST :} API REpresentational State Transfer.
C'est un type d'API permettant de communiquer avec un service web.
    \item \textbf{Base de données :} Entité informatique permettant de stocker des données.
    \item \textbf{Biophilie :} Du grec, littéralement "l'amour du vivant".
Ici, le fait d'installer dans les bâtiments des éléments issus du milieu naturel.
    \item \textbf{Changeset :} Une liste de changements.
    \item \textbf{Cluster :} Un groupe de machines.
    \item \textbf{Coworking :} Un espace de travail partagé regroupant une communauté de travailleurs.
    \item \textbf{CSV :} Comma-separated values. Un format de fichier permettant de stocker ou communiquer des données.
Chaque ligne est un élément de données dont les attributs sont séparés par des délimiteurs.
    \item \textbf{FAR :}
    \item \textbf{Fractale :} Forme géométrique dont le tracé est composé de plusieurs répétitions de lui-même.
    \item \textbf{Interopérabilité :} Le fait de pouvoir remplacer un élément par un autre sans accroc.
    \item \textbf{JSON :} JavaScript Object Notation. Format de fichier permettant de stocker ou communiquer des données hiérarchisées.
    \item \textbf{Mesh :}
    \item \textbf{Microgrid :}
    \item \textbf{Open Data :}
    \item \textbf{Oscillation endogène :} Horloge biologique.
    \item \textbf{Pattern :} Un motif dont la forme se répète successivement.
    \item \textbf{Serveur :} Un ordinateur accueillant un service et mis à disposition d'un réseau avec lequel il est capable de discuter.
    \item \textbf{Smart City :}
    \item \textbf{Smart Grid :}
    \item \textbf{STEP :} Station de Transfer d'Energie par Pompage.
Il peut s'agir d'un barrage dont on fait couler l'eau pour fabriquer de l'énergie, et en remonter quand on a besoin de stocker de l'énergie.
    \item \textbf{Thermodynamique :} Branche de la physique étudiant le mouvement de la chaleur.
    \item \textbf{WiMAX :}
\end{itemize}


Arc (Graphe) : Entité mathématique représentant un lien entre deux nœuds.
Arbre : Entité mathématique représentant un graphe avec un nœud parent et plusieurs enfants.
Arbre enraciné : Entité mathématique représentant un graphe avec un nœud parent et plusieurs enfants.
Commutateur : Appareil destiné à couper, à rétablir, à inverser le sens du courant électrique, ainsi qu'à le distribuer dans différents circuits
Contrôleur : Acteur qui
Courant continu :
Courent alternatif : Courent non continu.
Courent alternatif non-linéaire : Réseau électrique consommant de l'énergie.
Graphe : Entité mathématique représentant des relations avec nœud et arc.
Graphe Orienté : Entité mathématique représentant des relations avec nœud et arc orienté.
Nœud : Entité mathématique représentant un point d'un graphe.
Nœud fils : Nœud avec un parent.
Topologie : Description mathématique de la position de corps.
Intensité : Noté $U$
Voltage : Noté $V$
Transformateur :
Ensemble (mathématique) :
paradigme :

Hameçonnage : Méthode de hacking social jouant sur la crédulité des utilisateurs pour leurs soutirer des informations.