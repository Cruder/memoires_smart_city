\chapter*{Glossaire}

\begin{itemize}
    \item \textbf{Algorithme (mathématique) :} Une suite d'instructions ou d'opérations permettant de résoudre un problème.
    \item \textbf{Arbre enraciné (graphe) :} Graphe présentant une hiérarchie avec un nœud au sommet et des nœuds fils liés à lui, pouvant eux-mêmes être père d'autre nœuds.
    \item \textbf{Arc (graphe) :} Entité mathématique représentant un lien entre deux nœuds.
    \item \textbf{API (informatique) :} Application Programming Interface. Il s'agit d'une interface par laquelle
il est possible d'échanger des informations avec une entité informatique.
    \item \textbf{API REST (informatique) :} API REpresentational State Transfer.
C'est un type d'API permettant de communiquer avec un service web.
    \item \textbf{Base de données (informatique) :} Entité informatique permettant de stocker des données.
    \item \textbf{Biophilie :} Du grec, littéralement "l'amour du vivant".
Ici, le fait d'installer dans les bâtiments des éléments issus du milieu naturel.
    \item \textbf{Changeset (informatique) :} Une liste de changements entre plusieurs versions d'un lot de données.
    \item \textbf{Cluster (informatique) :} Un groupe de machines.
    \item \textbf{Commutateur (électrique) :} Appareil destiné à couper, rétablir, inverser le sens du courant électrique, ainsi qu'à le distribuer dans différents circuits.
    \item \textbf{Contrôleur :}
    \item \textbf{Courant alternatif (électrique) :} Circuit électrique dont le courant change de sens à intervalle de temps régulier.
    \item \textbf{Courant alternatif non linéaire (électrique) :}
    \item \textbf{Courant continu (électrique) :} Circuit électrique dont le courant circule toujours dans le même sens.
    \item \textbf{Coworking :} Un espace de travail partagé regroupant une communauté de travailleurs.
    \item \textbf{CSV (informatique) :} Comma-separated values. Un format de fichier permettant de stocker ou communiquer des données.
Chaque ligne est un élément de données dont les attributs sont séparés par des délimiteurs.
    \item \textbf{Ensemble (mathématique) :} Permet de désigner une collection d'objets. L'Ensemble des nombres entiers relatifs se note $Z$ par exemple.
    \item \textbf{FAR (informatique) :}
    \item \textbf{Fractale (mathématique) :} Forme géométrique dont le tracé est composé de plusieurs répétitions de lui-même.
    \item \textbf{Graphe (mathématique) :} Branche des mathématiques permettant de représenter des structures composées d'objets (nœuds) mis en relation les uns aux autres.
    \item \textbf{Graphe orienté (graphe) :} Dans un graphe orienté, les relations entre les objets (nœuds) sont des flèches définissant une direction.
    \item \textbf{Hameçonnage :} Méthode de hacking social jouant sur la crédulité des utilisateurs pour leurs soutirer des informations.
    \item \textbf{Intensité (électrique) :} Grandeur permettant de mesurer la vitesse du courant électrique dans un fil électrique, noté I, dont l'unité est l'ampère ($A$)
    \item \textbf{Interopérabilité :} Le fait de pouvoir remplacer un élément par un autre sans accroc.
    \item \textbf{JSON (informatique) :} JavaScript Object Notation. Format de fichier permettant de stocker ou communiquer des données hiérarchisées.
    \item \textbf{Mesh :}
    \item \textbf{Microgrid :} Réseau énergétique sans intelligence et incapable de communiquer, WIP
    \item \textbf{Nœud (graphe) :} Object d'un graphe.
    \item \textbf{Nœud fils (graphe) :} Object d'un graphe en arbre relié à un nœud du niveau supérieur.
    \item \textbf{Open Data (informatique) :} Collecter de la données pour la rendre publique et utilisable par des interfaces utilisateurs comme des applications.
    \item \textbf{Oscillation endogène :} Horloge biologique.
    \item \textbf{Paradigme (mathématique) :} Modèle que l'on se fait pour représenter une vision des choses.
    \item \textbf{Pattern (mathématique) :} Un motif dont la forme se répète successivement.
    \item \textbf{Serveur (informatique) :} Un ordinateur accueillant un service et mis à disposition d'un réseau avec lequel il est capable de discuter.
    \item \textbf{Smart City :} Ville intélligente capable de collecter des données de ses résidents pour optimiser le confort de vie.
    \item \textbf{Smart Grid :} Réseau énergétique intelligent et capable de communiquer, dont le but est d'optimiser les échanges.
    \item \textbf{STEP :} Station de Transfer d'Energie par Pompage.
Il peut s'agir d'un barrage duquel on fait couler l'eau pour fabriquer de l'énergie, et en remonter quand on a besoin de stocker de l'énergie.
    \item \textbf{Tension (électrique) :} Grandeur permettant de mesurer la circulation d'un champs électrique le long d'un circuit, noté $U$, dont l'unité est le volt ($V$).
    \item \textbf{Thermodynamique :} Branche de la physique étudiant le mouvement de la chaleur.
    \item \textbf{Topologie (mathématique) :} Description mathématique de la position de corps.
    \item \textbf{Transformateur (électrique) :} Appareil permettant de Transformer la tension et l'intensité électrique dans un circuit électrique alternatif.
    \item \textbf{Voltage (électrique) :} Anglicisme de la tension électrique, noté $U$, se mesure en volt (V).
    \item \textbf{WiMAX :}
\end{itemize}