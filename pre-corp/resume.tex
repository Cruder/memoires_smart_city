\chapter*{Résumé}

Dans l'objectif de satisfaire les attentes toujours plus importantes des politiques eco-responsables,
la recherche dans le secteur de l'énergie et la distribution de celle-ci nécessite de plus en plus d'automatisation.
Le besoin de données pour faire des prévisions réalistes, ainsi que la nécessité de micro-manager les différents noeuds
d'un réseau ont fait apparaître différents champs d'applications. Bien que les méthodes traditionnelles fonctionnent
pour de petites villes, les mégalopoles d'aujourd'hui et demain devront faire face à de nombreuses mutations.

L'énergie n'est plus distribuée de la même façon qu'il y a 20 ans, et il faut pouvoir à tout moment contrôler
le taux de saturation du réseau et l'équilibre offre/demande qui varie bien plus avec les énergies renouvelables.
De nombreuses problématiques sont a résoudre, telles que les pics de consommation la nuit, lorsque le Soleil laisse sa place
à la Lune, ou bien qu'une sécheresse assèche une STEP nécessaire au bon fonctionnement d'un pôle industriel.

Grâce à la détection des pannes, et l'amélioration des infrastructures citadines, le comportement des Hommes
s'en retrouve modifié, tout comme la 4G apporta une dimension sociale aux habitants de la ville, la 5G et les objets
connectés apporteront un nouveau canal de discussion qui permettra aux habitants de mieux comprendre la ville, et
à la ville de mieux comprendre ses habitants. Certaines villes aujourd'hui adaptent leurs offres de déplacement
collectifs en fonction des déplacements de ses habitants, demain ces offres pourront être dynamique en temps réel.

Tout ceci nous a conduit à initier des recherches sur les corrections en temps réel que l'informatique pourra
apporter, ainsi que les différentes technologies aujourd'hui mises en place pour l'utilisation des réseaux.