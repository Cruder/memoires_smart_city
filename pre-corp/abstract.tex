\chapter*{Abstract}

In order to meet the ever-increasing expectations of eco-responsible policies,
research in the energy sector and its distribution requires more and more automation.
The need for data to make realistic forecasts, as well as the need to micro-manage the different nodes
of a network have revealed different fields of application to this. Although traditional methods work
for small towns, today's and tomorrow's megacities will have to face to drastics changes.

Energy is no longer distributed in the same way as it was 20 years ago, and it must
be controled at any timethe saturation rate of the grid and the supply/demand balance,
which varies much more with renewable energies.
Many objectives have to be solved, such as peak consumption at night, when the sun leaves its place
to the moon, or a drought dries out a PHES necessary for the proper functioning of an industrial pole.

Thanks to the detection of breakdowns and the improvement of urban infrastructures, the behaviour of people
changed, just as 4G brought a social dimension to the inhabitants of the city, 5G and IoT
will provide a new channel for discussion that will allow residents to better understand the city, and
the city to better understand its inhabitants. Some cities are now adapting their travel offers
Depending on the movements of its inhabitants, tomorrow these offers will be dynamic in real time.

All this has led us to initiate research on real-time corrections that IT can make
to provide, as well as the different technologies currently in place for the use of networks.
