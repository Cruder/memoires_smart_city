\chapter{État de l’art}

\section{Le dérèglement climatique}
Nous pouvons faire l’analogie avec une application logicielle. Une règle courante est d’éviter l’optimisation prématurée, c’est à dire de ne pas optimiser tant que le besoin ne s’en fait pas ressentir. Nous risquerions de dépenser du temps et de l'énergie dans quelques chose qui peut-être ne fonctionne pas comme nous le voudrions.
Dans un premier temps, il faut faire en sorte que le système fonctionne correctement. On peut préférer la quantité face à la qualité. On ne cherche à optimiser cette méthode qu’ensuite, lorsque les ressources commencent à ne plus suffire ou qu’un besoin de remise à l’échelle se fait sentir. 

Aujourd’hui nous sommes en mesure de produire suffisamment d’énergie pour répondre aux besoin de la population. Mais le contexte environnementale nous impose désormais d’optimiser nos méthodes : épuisement des énergies fossiles, réchauffement climatique, augmentation de la population.  


\section{Augmentation des besoins en énergie}
\section{Guerre de l'énergie}
\section{Amélioration des performances informatique}
