\chapter*{Introduction}
\addcontentsline{toc}{chapter}{Introduction}

% [amorce]
Longtemps considéré comme la mesure de la puissance économique d'un pays,
la production énergétique est aujourd'hui pointé du doigt comme principale
cause de la pollution mondiale.

De plus en plus demandé, cette énergie doit être utilisée de manière à limiter les gâchis,
et par la même occasion réduire le coût de celle-ci.

% [présentation du sujet]

% [motivation]
L'énergie est un thème peu abordé à l'ESGI, malgré l'intervention de Mme Loto dans son cours de GreenIT, les
aspects de l'impact de l'informatique dans l'équilibre entre production et consommation n'est pas abordé
du cycle master. Nous avons donc choisi ce sujet afin de répondre à nos propres interrogations sur ce thème.

% [cadre théorique]
Les différents gouvernements sont aujourd'hui bien vus lors de la mise en place
d'énergies vertes comme l'éolien et le solaire. Cependant, ces énergies intermittentes ajoutent
une complexité supplémentaire à l'équilibre énergétique des réseaux.

Avec des guerres de plus en plus pesantes sur le cours de l'énergie, les états cherchent à s'émanciper
des énergies fossiles telles que le gaz et le pétrole pour ne pas être victime d'une hausse des prix.
Le coût de l'énergie a un lien direct avec l'innovation.


% [problématique]
En quoi les Smart Grid réduisent la perte de production d'énergie, et aident à la
prise de décision ?

% [méthodologie]
Afin de traiter le sujet et de répondre aux questionnements émis, une prise de contact a été faite
avec Issy Grid, projet d'ampleur de mise en place de Smart Grid en Ile de France.
Au cours d'une interview d'une heure, l'entreprise EMBIX à répondu à nos interrogations concernant les modes
de mise en place et l'impact de la Smart Grid sur les particuliers et les entreprises.
De nombreuses lectures d'articles et de thèses scientifiques auront été nécessaires
pour l'écriture de ce mémoire.

% [objectif]
Nous voudrions comprendre les tenants et aboutissements des technologies de Smart Grid, annoncés comme
alliant écologie et technologie.

% [annonce de plan]
En prenant principalement l'exemple de la France, nous verrons comment ces techniques s'allient avec
de nouvelles techniques de stockage.

Du monde d'aujourd'hui aux technologies de demain, nous détaillerons le fonctionnement des réseaux intelligents
appelés Smart Grid, son impact dans les villes ainsi que sa mise en place.

Ces réseaux qui ont pour but d'optimiser la production et réduire les erreurs, sont dépendants des différents cycles
mondiaux, que ce soit le cycle jours-nuit de la Terre, les marrées causées par la Lune, ou tout simplement l'Humain.
Une bonne connaissance de ces cycles ajoutent de la fiabilité aux prédictions de consommation et de production.

Une fois les besoins mis en relation avec la production, l'optimisation de la disposition
des producteurs peut être faite.
Le fait que chaque consommateur peut produire de l'énergie demande de prendre en compte l'aspect
multidirectionnel du réseau de distribution.

Enfin, nous verrons l'impact de la technologie sur ces mutations du cycle de production et de distribution d'énergie.
Le cas du Chili des années 1960 sera étudié pour en extraire les concepts fondamentaux des Smart City.
Le cycle de vie d'une information, de sa collecte à la prédiction future, tout en passant par le smartphone du citoyen.
