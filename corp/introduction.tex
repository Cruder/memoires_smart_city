\chapter*{Introduction}
\addcontentsline{toc}{chapter}{Introduction}

% [amorce]
Longtemps considéré comme la mesure de la puissance économique d'un pays,
la production énergétique est aujourd'hui pointé du doigt comme principale
cause de la pollution mondiale.

De plus en plus demandé, cette énergie doit être utilisé de manière à limiter les gâchis,
et par la même occasion réduire le coût de celle-ci.

% [présentation du sujet]

% [motivation]
L'énergie est un thème peu abordé à l'ESGI, malgré l'intervention de Mme Loto dans son cours de GreenIT, les
aspects de l'impacte de l'informatique dans l'équilibre entre production et consommation n'est pas abordé
du cycle master. Plusieurs questionnements nous ont interpellé et notre intérêt face au sujet
nous à fait choisir ce thème.

% [cadre théorique]
Les différents gouvernements sont aujourd'hui bien vu lors de la mise en place
d'énergie verte comme l'éolien et le solaire. Cependant, ces énergies intermittente ajoutent
une complexité supplémentaire à l'équilibre énergétique des réseaux.

Avec des guerres de plus en plus pesante sur le cours de l'énergie, les états cherchent à s'émancipé
des énergies fossiles tel que le gaz et le pétrole pour ne pas être victime d'une hausse des prix.
Le coût de l'énergie à un lien directe avec l'innovation.


% [problématique]
En quoi les smart grid réduisent la perte de production d'énergie, et aident à la
prise de décision ?

% [méthodologie]
Afin de traiter le sujet et de répondre aux questionnements émis, une prise de contacte à été fait
avec Issy Grid, projet d'ampleur de mise en place de smart grid en Ile de France.
Avec une interview d'une heure, l'entreprise MBIX à répondu à nos interrogations concernant les modes
de mise en place et l'impacte de la smart grid sur les particulier et les entreprises.
De nombreuses lectures d'articles et de thèses scientifiques aurons été nécessaire
pour l'écriture de ce mémoire.

% [objectif]
Nous voudrions comprendre les tenants et les aboutissement des technologies de smart grid, annoncé comme
alliant écologie et technologie.

% [annonce de plan]
En prenant principalement l'exemple de la France, nous verrons comment ces techniques s'allient avec
de nouvelles techniques de stockage.

Du monde d'aujourd'hui aux technologies de demain, nous détaillerons le fonctionnement des réseaux intelligent
appelé Smart Grid, son impacte dans les villes ainsi que sa mise en place.

Ces réseaux qui ont pour but d'optimiser la production et réduire les erreurs, sont dépendent des différents cycles
mondiaux, que ce soit le cycle jours nuit de la terre, les marrées causé par la lune, ou tout simplement l'Humain.
Une bonne connaissance de ces cycles ajoutent de la fiabilité aux prédictions de consommations et de production.

Une fois les besoins mis en relation avec la production, l'optimisation de la disposition
des producteurs peux être fait.
Le fait que chaque consommateur peux produire de l'énergie demande de prendre en compte l'aspect
multi-directionnel du réseau de distribution.

Enfin, nous verrons l'impacte de la technologie sur ces mutations du cycle de production et de distribution d'énergie.
Le cas du Chili des années 1960 sera étudié pour en extraire les concepts fondamentaux des smart city.
Le cycle de vie d'une donnée, de sa collecte à la prédiction future, tout en passant par le smartphone du citoyen.
