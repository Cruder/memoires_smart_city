\chapter{Cycles énergétiques}
\section{Cycle du soleil}

La terre tourne autour du soleil en 365 jours et 8 heures, mais tourne aussi sur elle même en 24 heures.
Ce cycle de 24 heures appelé le cycle jour-nuit est influé par l'orientation de la terre face au soleil,
en effet la terre a un angle de ?? degré avec le soleil.

Suivant quel pole est face au soleil, l'ensoleillement diffère avec des extrèmes aux poles ou la nuit ne tombe
qu'une fois par an pour une durée de 6 mois.

Plus on s'approche de l'équateur, moins cet effet est ressentis, ce qui à un impacte fort sur les technologies
accessible et utilisable par les différentes populations.

La rentabilité d'un panneau solaire est plus faible au pôles qu'a l'équateur.
En revanche, du fait des changements de température important au long de l'année, les régions proches des pôles
sont propices au vent et donc aux énergies éoliennes.



\section{Cycle circadien humain}
% cycles d’environ 24 heures
% Synchronisation par la lumiere du jour au niveaux de l’hypothamus
% le niveau de mélatonine sanguin est très faible le jour
% Plus la lumière diminue d’intensité, plus ce niveau augmente pour atteindre un degré maximal de sécrétion entre deux et quatre heures du matin

% Le cycle est different en fonction de l'age
% Newborn babies: They still don’t have a well formed circadian cycle, that’s the reason why they sleep on and off. As their sleep cycle gets more consolidated, daytime naps are less frequent and sleep is consolidated during the night.
% Children: Children need more sleep than adults, 11-13 hours per night between age 3-5 and 10-11 hours between 6 and 9 years old.
% Teenagers: Hormonal shifts change the circadian rhythm, making them go to bed later at night, and waking them up later in the morning.
% Adults: Adults needs between 7-9 hours of sleep every night. Some lifestyle choices, such as consumption of caffeine, stress and screen use during the evening disrupt our sleep cycle.


% Activitees pouvant de-regler le cycle :
% - Blue LED lightning, commonly emitted by screens, reduces melatonin production. This impacts our ability to fall asleep. We strongly encourage you to use blue light filters during the evening .
% - Travail de nuit

% Sources
% https://fondationsommeil.com/le-cycle-circadien/
% https://en.getmoona.com/blogs/mission-sleep/how-your-circadian-rhythm-influences-your-sleep

\includegraphics[scale=1.00]{media/circadien.png}

\section{Cycles long cours}
% Cette partie parlera des differents cycles terrestres prenant cours sur le long terme
% ex. Courants marins - Point chaud

Sur Terre, de nombreux cycles sont à des échelles plus grandes.

Les différents courants marins tel que le golf stream n'éxistaient pas il y a 10 000ans, et commencent à faiblir.

Les plaques tectonique déplacent les continent à raison de 5cm par an, c'est un changement infime
qui à produit l'archipel des Caraïbes, en effet un point chaud, résultant d'un chaleur du centre de la Terre,
reste statique et produit une nouvelle ile tous les 200 000ans % (retrouver la source)

Ces différent cycles impose à l'Homme de les prendre en compte dans leurs déploiement d'utilisation de l'énergie.

L'affaiblissement du Golf Stream rend inintéressant l'exploitation de celui-ci par des hydroliennes contrairement
à ce que les mesures des précédentes années montrait.

% Mettre une image de la chute de vitesse du golf stream

\section{Équilibre de la production}

Du fait de ces déséquilibres géographique, il n'est pas possible de compter sur la même source d'énergie partout
sur le globe, il va falloir équilibrer la production aux différentes sources locales.

Les régions insulaires sont riches en volcan et la géothermie est un moyen simple d'utiliser cette énergie.
Les régions proche des ( au dessus des tropiques ) profitent d'un anti-cyclone permanent et d'un ensoleillement
idéale pour la technologie solaire, à condition d'amélioré l'efficacité de celle-ci lors de fortes températures.

\section{Anticipation de la production}

Il est possible d'anticiper la production à condition de connaître toutes les sources d'énergies
exploité dans un cluster donné.

Si par exemple la zone est équipé de panneau solaires, on peux estimer sa production à l'aide de
l'ensoleillement et de la couverture nuageuse.

L'ensoleillement est facile à connaître, les satellites en orbite de la Terre arrivent à l'indiquer à la minute près
des mois à l'avance.
Il est plus difficile en revanche de prédire la couverture nuageuse de l'année du au chaos qu'est le climat.
