\chapter*{Conclusion}
\addcontentsline{toc}{chapter}{Conclusion}

La Smart Grid est un outil puissant.
Il nous permet d'aller très loin dans l'optimisation de la production, du transport,
du stockage et de la consommation d'énergie.
Essayer de produire au plus proche, de produire au bon moment, de stocker avec le moins de perte d'énergie possible.
Tout cela en s'adaptant au contexte du lieu en question en identifiant les méthodes
les plus adaptées pour tirer profit d'un biome particulier.
Permettre aux consommateurs de produire leur propre énergie, c'est de l'énergie en moins à transporter et à perdre en chaleur.

Le retour d'expérience d'IssyGrid nous a cependant appris qu'avant de mettre en place toutes ces spécificités,
il y avait moyen d'économiser une certaine quantité d'énergie avec de l'instrumentation
simple afin de détecter et remplacer les infrastructures défectueuses.


Le terme Smart Grid est très marketing. Nous sommes au clair sur les objectifs que cette infrastructure cherche à résoudre, mais pas toujours sur les moyens utilisés.
En réalité, Smart Grid et Smart City inspirent deux choses :
Optimisation énergétique dans une problématique environnementale
Confort de vie

Le projet Smart Ways de IssyGrid nous le rappelle bien. Assurer une couverture Wi-Fi et 4G dans un immeuble de travail, ou développer des applications dans ce même cadre ne répond pas à un problème d’optimisation de l'énergie, mais profite bien de l’excuse de la Smart Grid pour y trouver son chemin.
