\chapter*{Conclusion}
\addcontentsline{toc}{chapter}{Conclusion}

Ce mémoire avait pour objectif de mesurer l'efficacité des Smart Grid dans un contexte de changement
de paradigme de production énergétique en se demandant si les réseaux intélligents étaient efficaces,
ou une simple communication commerciale.

Une fois l'état de l'art fait, nous nous sommes attaqué aux différents types de réseaux intélligents,
et les avons confronté avec les données actuelles.

Ces analyses nous ont conforté dans l'idée que la Smart Grid était un outil puissant.
Il nous permet d'aller très loin dans l'optimisation de la production, du transport,
du stockage et de la consommation d'énergie.
Essayer de produire au plus proche, de produire au bon moment, de stocker avec le moins de perte
d'énergie possible.

Tout cela en s'adaptant au contexte du lieu en question en identifiant les méthodes
les plus adaptées pour tirer profit d'un biome particulier.
Permettre aux consommateurs de produire leur propre énergie,
c'est de l'énergie en moins à transporter et à perdre en chaleur.

Il convenait ensuite de comprendre pourquoi les Smart Grid étaient éfficaces.
L'analyse des différents cycles mondiaux démontrent bien le besoin de connaître le futur pour
mieux préparer le présent.

Le retour d'expérience d'IssyGrid nous a cependant appris qu'avant de mettre en place toutes ces spécificités,
il y avait moyen d'économiser une certaine quantité d'énergie avec de l'instrumentation
simple afin de détecter et remplacer les infrastructures défectueuses.

Le transport de cette énergie est elle aussi consommatrice d'énergie, réduire les distances est primordiale
pour réduire les pertes liées au transport et à la consommation.
Réduire les distances n'est pas la seule manière optimale de réduire la consommation,
les problématiques de charges impose de créer de nouvelles interconnexions entre consommateurs
d'une Smart Grid pour dé-saturer le réseau et être plus résilient aux pannes.

L'aspect Smart Grid ne serait rien sans la gestion de la données, de nombreux systèmes mis en place
dans le passé tels que Cybersyn influence notre manière de concevoir les applications d'aujourd'hui.
Le remplacement des capteurs avant leur fin de vie permet un support continu des différents
mécanismes de la Smart Grid.

L'Open Data est une grande opportunité économique et sociologique qui apporte de grandes avancées
dans l'usage urbain.

Malgré tout, l'informatisation reste un risque supplémentaire que les villes prennent pour améliorer
le confort de vie.
Avec tout ces capteurs, il devient plus facile pour un organisme malveillant de prendre contrôle d'une ville,
que ce soit à but économique ou idéologique.

Ce travail de mémoire se voulait exhaustif, mais il serait pertinent de procéder à des études plus approfondies
sur l'urbanisme, le transfert de chaleur, ou bien les bourses de l'énergies.
De nombreux champs des possibles ne sont pas abordés dans ce mémoire en raison d'un manque de compétences en
mathématiques ou en physique.
