\chapter*{Conclusion}
\addcontentsline{toc}{chapter}{Conclusion}

Ce mémoire avait pour objectif de mesurer l'efficacité des smart city dans un contexte de changement
de paradigme de production énergétique en se demandant si les réseaux intélligents étaient efficace,
où simple communication commerciale.

Une fois l'état de l'art fait, nous nous sommes attaquer aux différents types de réseaux intélligents,
et les avons confronté avec les données actuelles.

Ces analyses nous sont pousser vers l'état que la Smart Grid est un outil puissant.
Il nous permet d'aller très loin dans l'optimisation de la production, du transport,
du stockage et de la consommation d'énergie.
Essayer de produire au plus proche, de produire au bon moment, de stocker avec le moins de perte d'énergie possible
réduit les pertes.

Tout cela en s'adaptant au contexte du lieu en question en identifiant les méthodes
les plus adaptées pour tirer profit d'un biome particulier.
Permettre aux consommateurs de produire leur propre énergie, c'est de l'énergie en moins à transporter et à perdre en chaleur.

Le retour d'expérience d'IssyGrid nous avait cependant appris qu'avant de mettre en place toutes ces spécificités,
il y avait moyen d'économiser une certaine quantité d'énergie avec de l'instrumentation
simple afin de détecter et remplacer les infrastructures défectueuses.

Le terme Smart Grid est très marketing.
Nous sommes au clair sur les objectifs que cette infrastructure cherche à résoudre, mais pas toujours sur les moyens utilisés.
En réalité, Smart Grid et Smart City inspirent deux choses :
Optimisation énergétique dans une problématique environnementale et confort de vie.

Le projet Smart Ways de IssyGrid nous le rappelle bien.
Assurer une couverture Wi-Fi et 4G dans un immeuble de travail,
ou développer des applications dans ce même cadre ne répond pas à un problème
d’optimisation de l'énergie, mais profite bien de l’excuse de la Smart Grid pour y trouver son chemin.

Il convenait ensuite de comprendre pourquoi les smart grid sont éfficace.
L'analyse des différents cycles mondiaux démontrent bien le besoin de connaître le futur pour
mieux préparer le présent.

Le transport de cette énergie est elle aussi consommatrice d'énergie, réduire les distances est primordiale
pour réduire les pertes lié au transport et à la consommation.
Réduire les distances n'est pas la seule manière optimale de réduire la consommation,
les problématiques de charges impose de créer de nouvelles interconnexions entre consommateurs
d'un smart grid pour dé-saturer le réseaux et être plus résilient aux pannes.

L'aspect Smart City ne serais rien sans la gestion de la données, de nombreux systèmes mis en place
dans le passé tel que Cybersyn influence notre manière de concevoir les applications d'aujourd'hui.
Le remplacement des capteurs avant leurs fin de vie permet un support continue des différents
mécanismes de la smart grid.

L'Open Data est une grande opportunité économique et sociologique qui apporte de grande avancé
dans l'usage urbain.

Malgré tout, l'informatisation reste un risque supplémentaire que les villes prennent pour améliorer le confort de vie.
Avec tout ces capteurs, il deviens plus facile pour un organisme malveillant de prendre contrôle d'une ville,
que ce soit à but économique ou idéologique.

Ce travail de mémoire se voulais exhaustif, mais il serait pertinent de procéder à des études plus approfondie
sur l'urbanisme, le transfert de chaleur, ou bien les bourses de l'énergies.
De nombreux champs des possibles ne sont pas abordé dans ce mémoire du à un manque de compétence en
mathématique ou de physique.