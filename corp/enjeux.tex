\chapter{Enjeux}
\section{Descriptif de la smart grid}

% Smart grid

La smart grid est l'union de deux expertises, l'informatique et l'énergie.
Ce type réseau permet l'échange d'informations sur l'état du réseau.
Ces connaissances sont importantes pour pouvoir optimiser la distribution, faire du stockage 
ainsi que l'impacte écologique de l'homme.

L'un des intérêts de ce partage d'information est de pouvoir décentraliser la production d'énergie,
ainsi, tout point du réseau peut produire ou consommer son energie.
Cette complexité supplémentaire nécessite de nombreux ajustements et changements de paradigme.
L'un de ces changement est le coût de l'énergie, celle-ci varie déjà aujourd’hui, mais cette variation
devrait être d'autant plus importante que les énergies renouvelables prendront une place forte dans
le mix énergétique citadin. Une solution serait que chaque ville possède une bourse de l'énergie
locale qui atténuerait les variations, et profiterait aux personnes consommant en heures creuses.

% Micro grid

La microgrid est un autre type de réseau qui, contrairement à la smart grid, ne s'occupe pas de la communication.
Cette solution est envisagée plus sérieusement par de nombreuses communes pour des raisons de coût, en effet, ses
avantages restent nombreux tels que la gestion de différentes sources d'énergies opérée de manière parallèle,
ainsi qu'une fiabilisation du réseau déjà présent.

% https://www.researchgate.net/post/What_is_the_difference_between_a_microgrid_and_a_smartgrid
% A microgrid is an electrical system that includes multiple loads and distributed energy resources that can be operated in parallel with the 
% broader utility grid or a Small, independent power system. It Increased reliability with distributed generation, Increase efficiency 
% with reduced transmission length, and Easier integration of alternative energy sources.while 
% A smart grid is a modernized electrical grid that uses information and communications technology to gather and act on information, 
% such as information about the behaviors of suppliers and consumers, in an automated fashion to improve the efficiency, reliability, 
% economics, and sustainability of the production and distribution of electricity. Transmission and operations: wide‐area monitoring, 
% control and protection.


% Analyse des villes

\section{Villes connectées}
% Cette partie parlera de la partie technique de la mise en place de la smart grid
%   Ainsi que des différents problèmes que les villes actuelles ont.

\section{Connectivité et population}
% Cette partie parlera de la partie humaine de la connexion a la smart grid.
% ex. La 5G (6G) et les réseaux sociaux
%     Communication et report de problèmes

\section{Les mutations de la ville}



\section{Étude de cas - IssyGrid}
Le projet IssyGrid® a été initié en 2012 par Bouygues Immobilier dans la ville d'Issy-les-Moulineaux,
commune française dans le département des Hauts-de-Seine. Le projet a été fortement encouragé par le 
maire de la commune, André Santini. Mais aucun soutien d'ordre financier.

``La ville n’a pas mis un euro'' --  Éric Legale, directeur d’IssyMedia, chargé de la communication et 
de l’innovation de la ville.

Le projet n'a pas impliqué qu'un seul acteur, mais un consortium de dix compagnies : 
Bouygues Energies \& Services, Bouygues Immobilier, Bouygues Telecom, EDF, EMBIX, Enedis, 
Microsoft, Schneider Electric, Sopra Steria et Total.

L'expérience a duré 6 ans, de 2012 à 2018, et a un impact sur la vie de 5.000 habitants, 10.000 employés 
et 160.000 m2 de bureaux sur deux quartiers de la ville, Bords-de-Seine (quartier des affaires), 
puis Fort en 2015.

Le projet a fait face à plusieurs types de défis. 

Tout d'abord des défis techniques. 

Mais aussi un défi concernant la récolte et l'utilisation des données utilisateurs.

Un autre défi majeur fut la communication
% https://www.issy.com/issygrid

BILAN
2018
Des données non partagées
mais le maire felicite la reussiete du projet
ambitions 2020 une efficacité énergétique accrue de 20 \%, une réduction de 20 \% de l’empreinte carbone et une part de 20 % des énergies renouvelables 
: «Nous souhaitons capitaliser et ré-utiliser le savoir-faire acquis à Issy-les-Moulineaux pour la conception des futurs quartiers à énergie positive, notamment à « Issy Cœur de Ville » qui sera le 3e éco-quartier de la commune après celui du Fort et des Bords-de-Seine, mais aussi à Nanterre où est lancé un éco-quartier de 70 000 m² avec un smart grid axé autour de la chaleur », a conclu André Santini.


\section{Des projets en échec}

Lors de la VivaTech Paris 2019, Emmanuel BAVIERE (Société Générale), Erwan KERYER (KPMG) 
et Jérôme MONCEAUX (SPOON) ont pris la parole le vendredi 16 mai pour expliquer le phénomène autour des 
Smart Grid. Un terme fort a été employé durant cette prise de parole : 
``La Smart City ne marche pas''.

En se basant sur plusieurs exemples de projets en échec, divers arguments à ce propos ont été mis en 
avant : 
\begin{itemize}
    \item Des projets trop ambitieux ;
    \item Un manque de communication avec la population qui fait face à une puissance effrayante car mal comprise ;
    \item Les grands groupes ne savent pas comment gérer les tensions et désaccords avec les élus et la population ; 
    \item Un manque de formation de la population qui ne sait pas utiliser les nouvelles technologies ;
    \item Une population qui refuse de partager ses données pourtant nécessaires au bon fonctionnement d’une Smart Grid ;
    \item Le gouffre séparant les populations aisées des populations pauvres s'élargie.
\end{itemize}

Le projet Quayside de Toronto réunit la plupart de ces éléments. 
% Description du projet
