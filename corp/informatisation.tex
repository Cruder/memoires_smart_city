\chapter{Informatisation des infrastructures}

\section{Étude de Cybersyn}

1970 Chili
Salvador Allende au pouvoir
Régime Socialiste / Proche Communisme
Nationalisation
Intervenant de l'état dans les entreprises - Incompétences + Corruption
Fernando Flores propose une idée de Smart Grid
Idée qui viens de Stafford Beer, un cyberneticien
Theorie des systèmes viables

Cybernetique : Automatisme, Intéligence artificielle, Réseaux
Terme vieux

Projet Cybersyn => Cybernetic Synergy

Context : Pas encore internet - Arpanet née en 1969

Sujet du projet : La Planification
- L'état doit pouvoir planifier tout un pays (Régime Socialiste)
- Donner (illusion) du pouvoir aux ouvriers

Problèmes a résoudres :
- Bureaucratie
- Rétention d'information
- Famines


1 - Production et information venue des travailleurs
2 - Prises de decision - Data analysis et IA
3 - Simulation long terme et planification
4 - Arbitrage

Liberté = Interdependence

Pour simplifier la mise en place - Le model était récursif
12 niveaux de récursion

12 - Nation
11 - Gouvernement
10 - Économie
09 - Industrie
08 - Branche d'industrie
07 - Secteur d'industrie
06 - Sous-domaine
05 - Entreprise
04 - Département
03 - Atelier
02 - Équipe
01 - Travailleur


\section{Protocoles et Open Data}
% Communication entre services différents (voiture - infrastructure - éclairage - chauffage)
% Centralisation TaaS (Pilot Things)
\section{Gestion des catastrophes}
% Panne électrique par exemple
\section{Optimisation de la distribution énergétique}
\section{Cas d'étude d'une bourse de l'energie}
