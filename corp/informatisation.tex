\chapter{Informatisation des infrastructures}

\section{Étude de Cybersyn}

1970 Chili
Salvador Allende au pouvoir
Régime Socialiste / Proche Communisme
Nationalisation
Intervenant de l'état dans les entreprises - Incompétences + Corruption
Fernando Flores propose une idée de Smart Grid
Idée qui viens de Stafford Beer, un cyberneticien
Theorie des systèmes viables

Cybernetique : Automatisme, Intéligence artificielle, Réseaux
Terme vieux

Projet Cybersyn => Cybernetic Synergy

Context : Pas encore internet - Arpanet née en 1969

Sujet du projet : La Planification
- L'état doit pouvoir planifier tout un pays (Régime Socialiste)
- Donner (illusion) du pouvoir aux ouvriers

Problèmes a résoudres :
- Bureaucratie
- Rétention d'information
- Famines


1 - Production et information venue des travailleurs
2 - Prises de decision - Data analysis et IA
3 - Simulation long terme et planification
4 - Arbitrage

Liberté = Interdependence

Pour simplifier la mise en place - Le model était récursif
12 niveaux de récursion

12 - Nation
11 - Gouvernement
10 - Économie
09 - Industrie
08 - Branche d'industrie
07 - Secteur d'industrie
06 - Sous-domaine
05 - Entreprise
04 - Département
03 - Atelier
02 - Équipe
01 - Travailleur

Autonomie du systèmes tout en faisant partit d'un tout

Réseau telex

Formation :
- Formateurs
- Dessin animé
- Chansons

Utilité du projet (avant sa mort) :
- Connaitre l'état économique du pays (en quelques jours contre des mois)
- Pouvoir prévenir les crises (grève - sabotage)

Echec :

Projet lancé en 1971 - Coup d'état en 1973
20\% des entreprises connecté au réseau

Avant sa fin de vie, Sybersync n'intégra que le niveau 05 au niveau 09

Projet trop complexe - peu de gens le comprenaient
ouvriers jamais impliquer dans le processus

Technologie trop avangardiste

Reflexions sur le projet :

Projet ambitieux surtout pour son époque
Pouvant servire pour l'intéret general ou pour un dictateur en fonction de peu de critères

La surveillance et l'information sont le nerf du fonctionnement de ce genre de systèmes


Notes de l'interview de Eden Medina
https://www.youtube.com/watch?v=9qKoaQo9GTw

Il y avais 50 ordinateurs au Chili en 1971
4 ordinateurs de l'état
dont 1 du projet cybersyn

Le projet a imaginer un réseau national de plusieurs miliers d'ordinateurs
avec 1 seul

\section{Protocoles et Open Data}
% Communication entre services différents (voiture - infrastructure - éclairage - chauffage)
% Centralisation TaaS (Pilot Things)
\section{Gestion des catastrophes}
% Panne électrique par exemple
\section{Optimisation de la distribution énergétique}
\section{Cas d'étude d'une bourse de l'energie}
