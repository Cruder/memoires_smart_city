\chapter{Repartition des sources de prodution d'energies}
% \section{}
% \section{Reutilisation des sources de chaleurs}

L'emplacement des points de productions sont important dans la lute contre le gaspillage energetique.
Plusieurs points important entrent en ligne de compte, le cout du transport de cette energie ainsi que les pertes liee a ce mode de transport.
Peut importe la technologie utilise, les loies de la thermodynamique rendent impossible le transport d'energie gratuit.

\section{Perte d'energie en chaleure}

Prenant en compte la 1ere loi de la thermodynamique, lors de toute transformation, il y a conservation de l'énergie.
Ainsi, dans un espace clos, l'energie ne peut varier.
Prenons pour espace le cable qui fait transiter l'energie de la production a la consomation

Le transport de l'energie vas en transformer en challeur une partie de celle ci, due aux forces de frotement des elections sur les atomes.

Il est possible de reduire cette perte en utilisant des materiaux plus conducteur que d'autre, mais il n'est pas possible de l'annuler.
De plus, pour des raisons economique, ces materiaux supraconducteur ne peuvent etre deployer en masse du fait de leur cout, et des conditions necessaire a leur stabilitee
Le Diborure de magnésium $MgB_2$ est un des materiaux ayant la plus haute temperature critique avec 39 K (-234 C)

Aujourd'hui, la majorite des cables electrique mondiaux sont en cuivre ou en aluminium, on vas s'interesser plus particulierement au cuivre qui possede le meilleur rendement.

Pour calculer la perte d'energie dans un cable en cuivre, on peut utiliser la formule $Loss = I^2\times R$ avec $I$ pour l'intensite, exprimer en ampere, et $R$ en Ohm.
On peut exprimer les Ohms avec la forumule suivante : $R = \frac{V}{I}$

La formule etendue deviens donc

\begin{equation} \label{loss}
  \begin{aligned}
    Loss & = \frac{I^2\times V}{I} \\
         & = I \times V
  \end{aligned}
\end{equation}

En france, on utilise la norme \texttt{NF\_C18-510} pour classer les lignes electriques

\begin{table}[]
  \begin{tabular}{lll}
  Tension    & Courent Alternatif     \\
  Très basse & $U_n \leq 50V$         \\
  Basse      & $50V < U_n \leq 1kV$   \\
  Haute      & $1kV < U_n \leq 50kV$  \\
  Très Haute & $U_n > 50kV$
\end{tabular}
\end{table}


Comparons les differentes tensions pour une puissance donnee.

\lipsum