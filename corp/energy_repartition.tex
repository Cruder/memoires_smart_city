\chapter{Répartition des sources de production d'énergies}
% \section{}
% \section{Réutilisation des sources de chaleur}

L'emplacement des points de production sont importants dans la lutte contre le gaspillage énergétique.
Plusieurs points importants entrent en ligne de compte, comme le coût du transport de cette énergie ainsi que les pertes liées à ce mode de transport.
Peu importe la technologie utilisée, les lois de la thermodynamique rendent impossible le transport gratuit d'énergie.

\section{Perte d'énergie en chaleur}

Prenant en compte la 1ère loi de la thermodynamique, lors de toute transformation, il y a
conservation de l'énergie.
Ainsi, dans un espace clos, l'énergie ne peut varier.
Prenons pour espace le câble qui fait transiter l'énergie de la production à la consommation.

Le transport de l'énergie va transformer en chaleur une partie de celle-ci en raison
des forces de frottement des électrons sur les atomes.

Il est possible de réduire cette perte en utilisant des matériaux plus conducteurs que
d'autres, mais il n'est pas possible de l'annuler.
De plus, pour des raisons économiques, ces matériaux supraconducteurs ne peuvent être
déployés en masse du fait de leur coût, et des conditions nécessaires à leur stabilité.
Le Diborure de magnésium $MgB_2$ est un des matériaux ayant la plus haute température
critique avec 39 K (-234 C).

Aujourd'hui, la majorité des câbles électriques mondiaux sont en cuivre ou en aluminium,
on va s'intéresser plus particulièrement au cuivre qui possède le meilleur rendement.

Pour calculer la perte d'énergie dans un câble en cuivre, on peut utiliser la formule $Loss = I^2\times R$ avec $I$ pour l'intensité, exprimée en ampère, et $R$ en Ohm.
On peut exprimer les Ohms avec la formule suivante : $R = \frac{V}{I}$ avec V pour le voltage.

La formule étendue devient donc

\begin{equation} \label{loss}
  \begin{aligned}
    Loss & = \frac{I^2\times V}{I} \\
         & = I \times V
  \end{aligned}
\end{equation}

En france, on utilise la norme \texttt{NF\_C18-510} pour classer les lignes électriques

\begin{table}[h]
  \begin{tabular}{lll}
    Tension    & Courent Alternatif     \\
    Très basse & $U_n \leq 50V$         \\
    Basse      & $50V < U_n \leq 1kV$   \\
    Haute      & $1kV < U_n \leq 50kV$  \\
    Très Haute & $U_n > 50kV$
  \end{tabular}
\end{table}

% Comparons les différentes tensions pour une puissance donnée pour en obtenir.

Prenons plusieurs tensions pour une puissance donnée sur un cable d'1km de long.

Ces données vont nous donner :
\begin{itemize}
  \item Le diamètre du cable
  \item Le voltage perdu
  \item La chaleur généré
\end{itemize}

